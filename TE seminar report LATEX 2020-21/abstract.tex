%5.	Abstract: - Not more than 200 words
\chapter*{Abstract\markboth{Abstract}{Abstract}}
Image forgery is nothing but manipulating digital images so as to hide or change some useful information contained in the images. Images are considered as the most effective way to convey information and manipulating this information sometimes creates havoc.
The action of tampering images which is done either for fun or to give false evidence can result in a disaster.
It is done in such a way that it cannot be determined by the naked human eye so many people have implemented various types of machine learning algorithms which they have implemented with handcrafted features to determine different types of forgery and whether an image is forged or not. These algorithms are used to extract the digital signature of an image and used to differentiate whether an image has tampered or not.
In previous works, various techniques have been implemented for either fine or coarse image splicing whereas a technique dealing with both needs to be devised.\\
\vspace{0.2in}
%\begin{keyword}
%	TSBTC local, TSBTC global, Image classification, forgery detection, machine learning,WEKA(Waikato Environment for Knowledge Analysis).
%	\end{keyword}
% 
