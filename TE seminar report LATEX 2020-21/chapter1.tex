\chapter{Technical Keywords}
In an advent of social media platforms or digital content sharing platforms the digital data or images can be shared to the world with utmost ease. With an acute rise in social media platforms like messaging applications or photo/post sharing applications, world-wide spread of   information takes place within a few minutes. However, it can prove to be a boon or bane for an individual and society. Malefactors tend to use these platforms for their misdeeds, such as spreading fake information in a compelling manner over digital platforms.When it comes to multimedia information images, authenticity check and detection of its forging becomes difficult. Advanced Attempts of splicing images make it very difficult for the human eye to identify whether the image is real or fake.Even a few primitive tools fail to identify the authenticity of images. Digitization is an imperishable characteristic of the 21st century. A digital image is a widespread form of information and is exponentially increasing. Acquisition and alteration of image data are effortlessly accomplished with the help of image editing tools easily available in the public domain. Malicious practices of spreading fake data threaten the security and welfare of the digital world users. The fabricated images are not easily determined with visual inspection and thus require sophisticated technology for detecting forged images. With advancements in image tampering technologies such as GIMP, Pixlr, Photoshop, PhotoScape, etc. it becomes effortless even for an amateur to tamper images flawlessly. Image forgery detection is broadly classified into Active approaches and Passive approaches. Active approaches have predefined information or characteristics of the image. Digital watermarking and digital signature are 2 main active approaches that embed pre-processed information that can be extracted from the image. Passive approaches are comparatively harder to detect and have no predefined information. The images are raw and forgery can be detected solely by various feature extraction techniques. Image splicing is a type of Passive Image forgery wherein the image contains no prior information of its origin. Image splicing is a technique wherein a portion from one image is cropped and pasted onto another image. The cropped part may contain different texture and boundary characteristics than the host image. Thus local and global feature descriptors are extracted from the RGB colour channels of the image. Various machine learning classifiers are implemented for the classification of images into spliced and authentic.

\section{Problem Definition}
	
Image forgery detection using machine learning with a fusion of global and local thepade’s sbtc features.

\section{Goals and Objectives}
\begin{itemize}
	\item	To study and implement the combination of the existing algorithms like the random forest, random tree,	etc.
	\item Identifying whether image tampers or not.
	\item To do a Feature combination of global and local TSBTC nary.
	\end{itemize}
	
\section{Motivation}
Digital images become a significant resource of information in the digital world as they are the fastest means of information and medium of communication with the advent of digital

technology, image tampering has become very common. Image Forgery is used mainly for fake news, election and religious polarization, increasing the political power and influence, defaming someone, etc. The adverse effects of such forgery can be long-lasting thus it becomes necessary to deal with such situations so as to reduce the damage caused by such agendas. The use of Machine learning algorithms instead of manual detection can help in better detection of forged and non forged images.



\section{Scope of the work}
\begin{enumerate}
	\item	The scope of the project is to categorize images as forged or original using global and local feature vectors to train machine learning classifiers.
	\item The scope of the project is to improve the existing methodologies present in image forgery detection and increase the prediction accuracy.
\end{enumerate}

\section{Outcomes}
\begin{itemize}
	\item Finding Accuracies for local TSBTC nary.
	\item Finding Accuracies for global TSBTC nary.
	\item Doing Feature fusion of local and global TSBTC feature vectors which gives better accuracy for 2,3,4 nary.
\end{itemize}


\begin{figure}[h!]
	\centering
	\includegraphics[width=0.7\linewidth]{Logo}
	\caption[Fig. 1]{Logo for Pimpri Chinchwad College of Engineering}
	\label{fig:logo}
\end{figure}

\begin{table}[h!]
	\caption[List of students]{Comp Branch}
	\label{Table 1}
	\centering
\begin{tabular}{|c|c|c|c|}
	\hline
	Sr. No.& Name of the student  & Roll No.  & Branch  \\
	\hline
	1.& Apoorva & TECOB299  & COMPUTER  \\
	\hline
	2.& Shivam & TECOD499  & COMPUTER \\
	\hline
\end{tabular}
\end{table}

\begin{equation}
	\int\limits_{x = 0}^2 {{x^2} + {y^2}} dx
\end{equation}



